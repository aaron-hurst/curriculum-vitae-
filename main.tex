\documentclass[a4paper,11pt]{article}

\input{template/formatting}
%%%%%%%%%%%%%%%%%%%%
% Macros
%%%%%%%%%%%%%%%%%%%%
\newcommand{\lastupdated}{MMM YYYY}
\newcommand{\name}{Aaron Lauridsen Hurst}
\newcommand{\currenttitle}{Postdoctoral Fellow}
\newcommand{\employer}{Aarhus University, Denmark}
\newcommand{\email}{xxxx@gmail.com}
\newcommand{\phonenumber}{+45xxxxxxxx}

\newcommand{\linkedin}{https://www.linkedin.com/in/ahhurst}
\newcommand{\linkedinlabel}{ahhurst}
\newcommand{\github}{https://github.com/aaron-hurst}
\newcommand{\githublabel}{aaron-hurst}
\newcommand{\googlescholar}{https://scholar.google.dk/citations?user=e1xhNmEAAAAJ}
\newcommand{\googlescholarlabel}{Google Scholar}


\begin{document}

\input{template/header}





\section{Profile}\vspace{-0.3em}

\noindent
Aaron is a professional and dedicated engineer with strong competencies in problem-solving and communication, deep technical understanding in data compression and analytics and a drive to satisfy customers.
He holds BSc (2016) and Masters (2018) degrees in electronic engineering from The University of Western Australia, together with a PhD from Aarhus University (2024).
His current research as a postdoc focuses on data compression, big data analytics and sensor calibration, especially within the context of IoT systems.




\section{Key Skills}\vspace{-1.3em}

\begin{multicols}{2}
    \begin{itemize}[topsep=0pt,itemsep=-0.3em]
        \item Software development, including scripting, backend, OOP and REST APIs
        \item Data analytics and predictive modelling, including with natural language
        \item Software testing and TDD
        \item Research, problem-solving, communication

        \item \textbf{Programming languages:} Python, C/C++
        \item \textbf{Frameworks/libraries:} Flask, Scikit-learn, Pandas, Matplotlib, Numpy, nltk
        \item \textbf{Tools:} VS Code, Git, Jupyter, PostgreSQL, MySQL, Postman
        \item \textbf{Languages:} English (native), Danish (fluent)
    \end{itemize}
\end{multicols}
\vspace{-0.7em}





\section{Experience}

\experience{Postdoctoral Fellow}{Aarhus University}{Nov 2023--Present}{
    \item Investigating the application of state-of-the-art data compression and analytics research to practical and commercial contexts, including through collaboration with Aarhus University Department of Biology and Villanova University.
    \item Supporting project management and funding application activities.
    \item Travelled to China with financial support from \textbf{Thomas B. Triges Fond} to presented my research at a major international databases conference .
}

\experience{PhD Fellow}{Aarhus University}{Nov 2020--Nov 2023}{
    \item Investigated the joint optimisation of lossless data compression and data analytics in IoT applications.
    \item Developed methods and software for high-efficiency data analytics, prioritising storage, accuracy and latency.
    \item Research visit to UCSD to work with Professor Tara Javidi on compression for federated learning fully funded by an \textbf{EliteForsk-rejsestipendium} (2023) awarded by The Danish Ministry of Higher Education and Science.
}

\experience{Data Scientist}{IBM Australia}{Feb 2018--Jul 2020}{
    \item Led a team of developers in building a decision-support web application for a large mining company.
    \item Established data science processes and software for predictive maintenance at a state water utility.
    \item Developed natural language models to help clients leverage non-structured data in business decision making.
}

\experience{Student Rail Engineer}{Public Transport Authority, Western Australia}{Dec 2016--Feb 2017}{
    \item Assisted with documentation management, asset reliability assessments and design verifications.
}

\experience{Research Intern}{Pawsey Supercomputing Centre}{Dec 2015--Feb 2016}{
    \item Investigating novel magneto-electronic devices using simulations run on Pawsey's supercomputers.
}

\experience{Tutor/Class Facilitator}{The University of Western Australia}{Feb 2016--Nov 2017}{
    \item Facilitated classes of 10--40 students in statistics, project management and semiconductor materials.
}

\vspace{0.9em}





\section{Education}

\experience{Master of Professional Engineering}{The University of Western Australia}{Feb 2016--Dec 2017}{
    \item Specialisation: \textbf{Electrical \& Electronic Engineering} with a focus on software.
    \item Thesis: developed a computer vision system for a physical autonomous vehicle testbed in C++. Results led to a co-authored conference paper.
    \item Achieved the highest weighted average mark in my cohort of 298 students (\textbf{88.6/100}, GPA: 6.94/7).
    \item Five prizes for top marks in individual courses.
}

\experience{Bachelor of Science}{The University of Western Australia}{Feb 2013--Dec 2015}{
    \item Majors: \textbf{Engineering Science} (Electronic) \& \textbf{Management}
    \item Achieved the highest weighed average mark in cohort (\textbf{90.6/100}, GPA: 6.96/7).
    \item Selected to give the valedictory address at my bachelor graduation ceremony.
    \item Rio Tinto Undergraduate Scholarship in Geology and/or Engineering Science \textcolor{subduedcolour}{(for academic achievement).}
    \item Tesla Medal in Electronic Materials and Devices \textcolor{subduedcolour}{(highest mark in my cohort).}
}





%\section{Publications}
%
%\noindent
%\begin{tabular}{p{0.05\textwidth}p{0.9\textwidth}}
%    \textbf{2024} & \textbf{A. Hurst}, D. E. Lucani, and Q. Zhang, “PairwiseHist: Fast, Accurate, and Space-Efficient Approximate Query Processing with Data Compression,” \textit{International Conference on Very Large Data Bases} (VLDB), 2024. \\
%
%     & \textbf{A. Hurst}, D. E. Lucani, and Q. Zhang, “GreedyGD: Enhanced Generalized Deduplication compression for IoT applications,” \textit{IEEE Transactions on Industrial Informatics}, 2024. \\
%
%    \textbf{2022} & \textbf{A. Hurst}, D E. Lucani, I. Assent, and Q. Zhang, “Glean: Generalized deduplication enabled approximate edge analytics,” \textit{IEEE Internet of Things Journal}, 2022. \\
%
%    \textbf{2021} & \textbf{A. Hurst}, D. E. Lucani, I. Assent, and Q. Zhang, ``Direct analytics of generalized deduplication compressed IoT data,'' \textit{IEEE Global Communications Conference} (GLOBECOM), 2021. \\
%
%    \textbf{2018} & R. Barker, \textbf{A. Hurst}, R. Shrubsall, G. M. Hassan and T. French, "A Low-Cost Hardware-in-the-Loop Agent-Based Simulation Testbed for Autonomous Vehicles," \textit{IEEE/ASME International Conference on Advanced Intelligent Mechatronics}, 2018. \\
%
%    \textbf{2017} & \textbf{A. Hurst}, J. A. Izaac, F. Altaf, V. Baltz, P. J. Metaxas, “Reconfigurable magnetic domain wall pinning using vortex-generated magnetic fields,” \textit{Applied Physics Letters}, 2017. \\
%\end{tabular}





%\section{Accomplishments}
%
%\noindent
%\begin{tabular}{lp{0.87\textwidth}}
%    \textbf{2023} & EliteForsk-rejsestipendium. \textcolor{subduedcolour}{The Danish Ministry of Higher Education and Science awards up to 20 stipends each year of 200,000 DKK to support PhD students undertaking international research visits. Candidates must be selected and proposed by their institute.} \\
%
%    \textbf{2022} & Passed Prøve i Dansk 3 (PD3) with top marks across all disciplines. \textcolor{subduedcolour}{This is the highest level standard Danish exam and a pre-requisite for certain permanent visas and citizenship.}\\
%
%    \textbf{2019} & Industrial Products Insights and Solutions (Bronze). \textcolor{subduedcolour}{Awarded by IBM based on experience and contributions to clients within industrial sectors.} \\
%
%    \textbf{2016} & Institution of Engineering and Technology Prize in Communication Systems. \textcolor{subduedcolour}{Awarded to the student who achieved the highest course grade. AU\$300 and one year membership.} \\
%
%    & MRX Technologies Prize in Control Engineering. \textcolor{subduedcolour}{Awarded to the student who achieved the highest course grade. AU\$1.000.} \\
%
%    & MRX Technologies Prize in Digital and Embedded Systems. \textcolor{subduedcolour}{Awarded to the student who achieved the highest course grade. AU\$1.000.} \\
%
%    & Monadelphous Prize in Project Management and Engineering Practice,. \textcolor{subduedcolour}{Awarded to the student who achieved the highest course grade. AU\$1.500.} \\
%
%    & Programmed Professionals Prize in Circuits and Electronic Systems. \textcolor{subduedcolour}{Awarded to the student who achieved the highest course grade. AU\$2.000.} \\
%
%    \textbf{2015} & Rio Tinto Undergraduate Scholarship in Geology and/or Engineering Science for academic achievement. \textcolor{subduedcolour}{Awarded to two students annually based on academic performance. AU\$6.000 and invitation to two masterclass workshops.} \\
%
%    & Selected to give the valedictory address at my bachelor graduation ceremony. \\
%
%    & Tesla Medal in Electronic Materials and Devices. \textcolor{subduedcolour}{Awarded to the student who achieved the highest course grade. % ENSC3014 Electronic Materials and Devices.
%    Cash prize and a physical medal.} \\
%
%    \textbf{2014} & John \& Robin de Laeter Scholarshop. \textcolor{subduedcolour}{Awarded for restoring science education exhibits at the Gravidt Discovery Centre museum, Gingin, Western Australia.}
%\end{tabular}



\end{document}

