\documentclass[a4paper,11pt]{article}

\input{template/formatting}
%%%%%%%%%%%%%%%%%%%%
% Macros
%%%%%%%%%%%%%%%%%%%%
\newcommand{\lastupdated}{MMM YYYY}
\newcommand{\name}{Aaron Lauridsen Hurst}
\newcommand{\currenttitle}{Postdoctoral Fellow}
\newcommand{\employer}{Aarhus University, Denmark}
\newcommand{\email}{xxxx@gmail.com}
\newcommand{\phonenumber}{+45xxxxxxxx}

\newcommand{\linkedin}{https://www.linkedin.com/in/ahhurst}
\newcommand{\linkedinlabel}{ahhurst}
\newcommand{\github}{https://github.com/aaron-hurst}
\newcommand{\githublabel}{aaron-hurst}
\newcommand{\googlescholar}{https://scholar.google.dk/citations?user=e1xhNmEAAAAJ}
\newcommand{\googlescholarlabel}{Google Scholar}


\begin{document}

\input{template/header}





\section{Profil}\vspace{-0.3em}

Aaron er en professionel og dedikerede forsker med stærk kompetencer inden for problemløsning og kommunikation, dyb teknisk forståelse i datakomprimering og dataanalyse og utrættelige nysgerrighed.
Har holder Bachelor of Science (2016) og Masters (2018) grader i elektronikingeniør fra The University of Western Australia, samt en ph.d. grad fra Aarhus Universitet (2024).
Hans nuværende forskning som postdoctoral fellow handler om datakomprimering, Big Data analyse, federated learning og sensorkalibrering, især indenfor Internet of Things.




\section{Relevante Evne}\vspace{-1.3em}

\begin{multicols}{2}
    \begin{itemize}[topsep=0pt,itemsep=-0.3em]
        \item Dataanalyse og forudseende modelling, inkl. med naturlig sprog
        \item Software udviking, inlk. scripting, backend, OOP og REST APIs
        \item Software testing og TDD
        \item Forskning, problemløsning, kommunikation

        \item \textbf{Programmeringsprog:} Python, C/C++
        \item \textbf{Frameworks/libraries:} Flask, Scikit-learn, Pandas, Matplotlib, Numpy, nltk
        \item \textbf{Værktøj:} VS Code, Git, Jupyter, PostgreSQL, MySQL, Postman
        \item \textbf{Sprog:} English (native), Danish (fluent)
    \end{itemize}
\end{multicols}
\vspace{-0.7em}




\section{Erhvervserfaring}


\experience{Postdoctoral Fellow}{Aarhus University}{nov 2023--nu}{
    \item Undersøger anvendelse af grundlæggende datakomprimering of dataanalyse forskning i praktiske og kommercielle sammenhænge, blandt anden gennem samarbejde med Aarhus Universitets Institut for Biologi og Force Technologies.
    \item Projektledelse og ansøgninger for finansiering/støtte.
    \item Rejste til Kina, hvor jeg fremlagde min forskning på en stor international konference. Rejsen var støttet af en legat, jeg fik fra \textbf{Thomas B. Triges Fond}.
}

\experience{PhD Fellow}{Aarhus University}{nov 2020--nov 2023}{
    \item Undersøgte fælles optimering af tamsfri datakomprimering og dataanalyse i IoT anvendelser.
    \item Udviklede metoder og software for højeffektivitet datanalyse med prioritering på lav lagerplads forbrug, høj nøjagtighed og kort latenstid.
    \item Udenlandsk forsnkingsophold hos UCSD for at arbejde med prof. Tara Javidi på ny datakomprimering løsninger til federated learning støttet af en \textbf{EliteForsk-rejsestipendium} (2023) uddelt af Uddannelses- og Forskningsministeriet.
}

\experience{Data Scientist}{IBM Australia}{feb 2018--jul 2020}{
    \item Ledte et hold af udviklere igennem at bygge en webapplikation for en stor kunder.
    \item Etablerede data science-processer og software til forudsigende vedligeholdelse hos et vandforsyningsselskab.
    \item Udviklede sprogmodeller til at hjælpe kunder med at udnytte deres ikke-strukturerede data i beslutninger.
}

\experience{Studerenterjernbaneingeniør}{Public Transport Authority, Western Australia}{dec 2016--feb 2017}{
    \item Hjulpet med dokumentationsstyring, vurderinger af aktivers pålidelighed og designverifikationer.
}

\experience{Forskningsintern}{Pawsey Supercomputing Centre}{dec 2015--feb 2016}{
    \item Undersøgte nye magneto-electroniske enheder, ved hjælp at simuleringer på Pawseys supercomputers. %, resulting in one publication.
}

\experience{Underviser}{The University of Western Australia}{feb 2016--nov 2017}{
    \item Ledte undervisningssessioner for 10-40 studerende i statistik, projektledelse og halvledermaterialer.
}

\vspace{-0.9em}





\newpage

\section{Uddannelse}

\experience{Master of Professional Engineering}{The University of Western Australia}{feb 2016--dec 2017}{
    \item Specialisering indenfor elektronikingeniør.
    \item Speciale: udviklede et computersyn system for en fysisk selvkørende bil testsystem i C++. Resultater ledt til en konference artikel.
    \item Opnået den højeste gennemsnitskarakter i min årgang på 298 studerende (88.6 / 100; GPA 6.94 / 7).
    \item Fem priser for højeste karakter i bestemte fag.
}

\experience{Bachelor of Science}{The University of Western Australia}{feb 2013--dec 2015}{
    \item Specialisering: elektronikingeniør og management
    \item Opnået den højeste gennemsnitskarakter i min årgang (90.6 / 100; GPA: 6.96 / 7).
    \item Udvalgt til at holde den ``valedictory address'' til min dimission.
    \item Rio Tinto Undergraduate Scholarship in Geology og/eller Engineering Science.
    \item Tesla Medal in Electronic Materials and Devices.
}





%\section{Publications}
%
%\noindent
%\begin{tabular}{p{0.05\textwidth}p{0.9\textwidth}}
%    \textbf{2024} & \textbf{A. Hurst}, D. E. Lucani, and Q. Zhang, “PairwiseHist: Fast, Accurate, and Space-Efficient Approximate Query Processing with Data Compression,” \textit{International Conference on Very Large Data Bases} (VLDB), 2024. \\
%
%     & \textbf{A. Hurst}, D. E. Lucani, and Q. Zhang, “GreedyGD: Enhanced Generalized Deduplication compression for IoT applications,” \textit{IEEE Transactions on Industrial Informatics}, 2024. \\
%
%    \textbf{2022} & \textbf{A. Hurst}, D E. Lucani, I. Assent, and Q. Zhang, “Glean: Generalized deduplication enabled approximate edge analytics,” \textit{IEEE Internet of Things Journal}, 2022. \\
%
%    \textbf{2021} & \textbf{A. Hurst}, D. E. Lucani, I. Assent, and Q. Zhang, ``Direct analytics of generalized deduplication compressed IoT data,'' \textit{IEEE Global Communications Conference} (GLOBECOM), 2021. \\
%
%    \textbf{2018} & R. Barker, \textbf{A. Hurst}, R. Shrubsall, G. M. Hassan and T. French, "A Low-Cost Hardware-in-the-Loop Agent-Based Simulation Testbed for Autonomous Vehicles," \textit{IEEE/ASME International Conference on Advanced Intelligent Mechatronics}, 2018. \\
%
%    \textbf{2017} & \textbf{A. Hurst}, J. A. Izaac, F. Altaf, V. Baltz, P. J. Metaxas, “Reconfigurable magnetic domain wall pinning using vortex-generated magnetic fields,” \textit{Applied Physics Letters}, 2017. \\
%\end{tabular}





%\section{Accomplishments}
%
%\noindent
%\begin{tabular}{lp{0.87\textwidth}}
%    \textbf{2023} & EliteForsk-rejsestipendium. \textcolor{subduedcolour}{The Danish Ministry of Higher Education and Science awards up to 20 stipends each year of 200,000 DKK to support PhD students undertaking international research visits. Candidates must be selected and proposed by their institute.} \\
%
%    \textbf{2022} & Passed Prøve i Dansk 3 (PD3) with top marks across all disciplines. \textcolor{subduedcolour}{This is the highest level standard Danish exam and a pre-requisite for certain permanent visas and citizenship.}\\
%
%    \textbf{2019} & Industrial Products Insights and Solutions (Bronze). \textcolor{subduedcolour}{Awarded by IBM based on experience and contributions to clients within industrial sectors.} \\
%
%    \textbf{2016} & Institution of Engineering and Technology Prize in Communication Systems. \textcolor{subduedcolour}{Awarded to the student who achieved the highest course grade. AU\$300 and one year membership.} \\
%
%    & MRX Technologies Prize in Control Engineering. \textcolor{subduedcolour}{Awarded to the student who achieved the highest course grade. AU\$1.000.} \\
%
%    & MRX Technologies Prize in Digital and Embedded Systems. \textcolor{subduedcolour}{Awarded to the student who achieved the highest course grade. AU\$1.000.} \\
%
%    & Monadelphous Prize in Project Management and Engineering Practice,. \textcolor{subduedcolour}{Awarded to the student who achieved the highest course grade. AU\$1.500.} \\
%
%    & Programmed Professionals Prize in Circuits and Electronic Systems. \textcolor{subduedcolour}{Awarded to the student who achieved the highest course grade. AU\$2.000.} \\
%
%    \textbf{2015} & Rio Tinto Undergraduate Scholarship in Geology and/or Engineering Science for academic achievement. \textcolor{subduedcolour}{Awarded to two students annually based on academic performance. AU\$6.000 and invitation to two masterclass workshops.} \\
%
%    & Selected to give the valedictory address at my bachelor graduation ceremony. \\
%
%    & Tesla Medal in Electronic Materials and Devices. \textcolor{subduedcolour}{Awarded to the student who achieved the highest course grade. % ENSC3014 Electronic Materials and Devices.
%    Cash prize and a physical medal.} \\
%
%    \textbf{2014} & John \& Robin de Laeter Scholarshop. \textcolor{subduedcolour}{Awarded for restoring science education exhibits at the Gravidt Discovery Centre museum, Gingin, Western Australia.}
%\end{tabular}



\end{document}

