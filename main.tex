\documentclass[a4paper,11pt]{article}

%%%%%%%%%%%%%%%%%%%%
% Packages
%%%%%%%%%%%%%%%%%%%%
\usepackage[english]{babel}
\usepackage[T1]{fontenc}
\usepackage{cmbright}  % European Computer Modern Bright font
\usepackage{graphicx}  % Required for inserting images
\usepackage[margin=2cm]{geometry}
\usepackage{nopageno}
\usepackage{xcolor}
\usepackage{tikz}
\usepackage{tikzpagenodes}
\usepackage{fancyhdr}
\usepackage{enumitem}
\usepackage{booktabs}
\usepackage[defaultlines=2,all]{nowidow}
\usepackage[skip=0em]{parskip}
\usepackage{multicol}
\usepackage[hidelinks]{hyperref}

\usetikzlibrary{backgrounds}  % allows framed option for tikzpicture

%%%%%%%%%%%%%%%%%%%%
% Colours
%%%%%%%%%%%%%%%%%%%%
\definecolor{accentcolour}{HTML}{359EBF}
\definecolor{subduedcolour}{HTML}{999999}

%%%%%%%%%%%%%%%%%%%%
% Section headers
%%%%%%%%%%%%%%%%%%%%
\renewcommand{\section}[1]{\vspace{1.7em}
\begin{tikzpicture}[remember picture,text depth=0.25em]
    \node[inner sep=0] (title) {\Large \textbf{#1}};
    \draw[thick, accentcolour] (title.east)+(0.5em,0) -- (title -| current page text area.east);
\end{tikzpicture}
\vspace{-0.4em}
}

%%%%%%%%%%%%%%%%%%%%
% Experience items
%%%%%%%%%%%%%%%%%%%%
\newcommand{\experience}[4]{
\begin{tikzpicture}[remember picture,inner sep=0,text depth=0.25em]
    \node[] (role) {\textbf{#1}, #2};
    \node[anchor=south east, subduedcolour] (period) at (role.south -| current page text area.east) {\footnotesize #3};
\end{tikzpicture}
\begin{itemize}[topsep=0pt,itemsep=-0.3em]
    {\small #4 }
\end{itemize}
\vspace{0.9em}
}

%%%%%%%%%%%%%%%%%%%%
% Headers & footers
%%%%%%%%%%%%%%%%%%%%
\fancyhf{}
\renewcommand{\headrulewidth}{0pt}
\fancyfoot[C]{\small \textcolor{subduedcolour}{\lastupdated}}
\pagestyle{fancy}

%%%%%%%%%%%%%%%%%%%%
% Other Layout Details
%%%%%%%%%%%%%%%%%%%%
\renewcommand{\arraystretch}{1.4}
\newcommand{\mellemrum}{\hspace{0.1cm}| \hspace{0.1cm}}

%%%%%%%%%%%%%%%%%%%%
% Macros
%%%%%%%%%%%%%%%%%%%%
\newcommand{\lastupdated}{June 2024}
\newcommand{\name}{Aaron Lauridsen Hurst}
\newcommand{\currenttitle}{Postdoctoral Fellow}
\newcommand{\employer}{Aarhus University, Denmark}
\newcommand{\email}{xxxx@gmail.com}
\newcommand{\phonenumber}{+45xxxxxxxx}

\newcommand{\linkedin}{https://www.linkedin.com/in/ahhurst}
\newcommand{\linkedinlabel}{ahhurst}
\newcommand{\github}{https://github.com/aaron-hurst}
\newcommand{\githublabel}{aaron-hurst}
\newcommand{\googlescholar}{https://scholar.google.dk/citations?user=e1xhNmEAAAAJ}
\newcommand{\googlescholarlabel}{Google Scholar}


\begin{document}

\begin{center}
    {\huge\textbf{\name}} \\ \vspace{0.2em}
    \currenttitle{} \vline{} \employer{} \\ \vspace{0.5em}

    % Email
    \href{mailto:\email}{
        \begin{tikzpicture}[inner sep=0.2em, text depth=0.2em]
            \node[] (icon) at (0,0) {\includegraphics[height=11pt]{icons/email.png}};
            \node[anchor=west, accentcolour] (label) at (icon.east) {\email};
        \end{tikzpicture}
    }
    % Phone
    \href{tel:\phonenumber}{
        \begin{tikzpicture}[inner sep=0.2em, text depth=0.2em]
            \node[] (icon) at (0,0) {\includegraphics[height=11pt]{icons/phone.png}};
            \node[anchor=west, accentcolour] (label) at (icon.east) {\phonenumber};
        \end{tikzpicture}
    }
    % LinkedIn
    \href{\linkedin}{
        \begin{tikzpicture}[inner sep=0.2em, text depth=0.2em]
            \node[] (icon) at (0,0) {\includegraphics[height=11pt]{icons/linkedin.png}};
            \node[anchor=west, accentcolour] (label) at (icon.east) {\linkedinlabel};
        \end{tikzpicture}
    }
%    % Personal website
%    \href{\personalwebsite}{
%        \begin{tikzpicture}[inner sep=0.2em, text depth=0.2em]
%            \node[] (icon) at (0,0) {\includegraphics[height=11pt]{icons/website.png}};
%            \node[anchor=west, accentcolour] (label) at (icon.east) {\websitelabel};
%        \end{tikzpicture}
%    }
    % Google Scholar
    \href{\googlescholar}{
        \begin{tikzpicture}[inner sep=0.2em, text depth=0.2em]
            \node[] (icon) at (0,0) {\includegraphics[height=11pt]{icons/googlescholar-bw.png}};
            \node[anchor=west, accentcolour] (label) at (icon.east) {\googlescholarlabel};
        \end{tikzpicture}
    }
%    % GitHub
%    \href{\github}{
%        \begin{tikzpicture}[inner sep=0.2em, text depth=0.2em]
%            \node[] (icon) at (0,0) {\includegraphics[height=11pt]{icons/github.png}};
%            \node[anchor=west, accentcolour] (label) at (icon.east) {\githublabel};
%        \end{tikzpicture}
%    }
\end{center}





\section{Profil}\vspace{-0.5em}

\noindent
Aaron er en professionel og grundig forsker med dyb forståelse indenfor datakomprimering og dataanalyse og tidligere erhvervserfaring som data scientist hos IBM Australia.
Han holder en Bachelor of Science (2016) og kandidat i elektronikingeniør (2018) fra The University of Western Australia, samt en ph.d. grad fra Aarhus Universitet (2024).
Hans nuværende forskningsinteresse drejer sig om datakomprimering, big data analyse og federated learning, især med fokus på IoT systemer.





\section{Erhvervserfaring}

\experience{Postdoctoral Fellow}{Aarhus Universitet}{november 2023--nu}{
    \item Fokus på at anvende nye teknologier fra egne forskning i kommercielle sammenhænge.
    \item Samarbejdsprojekter med blandt andre Institut for Biologi og Villanova University.
}

\experience{PhD Fellow}{Aarhus University}{november 2020--november 2023}{
    \item Undersøgte fælles optimering af tamsfri datakomprimering og dataanalyse i IoT anvendelser.
    \item Udviklede metoder og software for højeffektivitet datanalyse med prioritering på lav lagerplads forbrug, høj nøjagtighed og kort latenstid.
    \item Udenlandsk forsnkingsophold hos UCSD for at arbejde med prof. Tara Javidi på ny datakomprimering løsninger til federated learning.
}

\experience{Data Scientist}{IBM Australia}{februar 2018--juli 2020}{
    \item Ledte et hold af udviklere igennem at bygge en webapplikation for en stor kunder.
    \item Etablerede data science-processer og software til forudsigende vedligeholdelse hos et vandforsyningsselskab.
    \item Udviklede sprogmodeller til at hjælpe kunder med at udnytte deres ikke-strukturerede data i beslutninger.
}

\experience{Studerenterjernbaneingeniør}{Public Transport Authority, Western Australia}{december 2016--februar 2017}{
    \item Hjulpet med dokumentationsstyring, vurderinger af aktivers pålidelighed og designverifikationer.
}

\experience{Forskningsintern}{Pawsey Supercomputing Centre}{december 2015--februar 2016}{
    \item Undersøgte nye magneto-electroniske enheder, ved hjælp at simuleringer på Pawseys supercomputers. %, resulting in one publication.
}

\experience{Tutor}{The University of Western Australia}{februar 2016--november 2017}{
    \item Ledte undervisningssessioner for 10-40 studerende i statistik, projektledelse og halvledermaterialer.
}





\section{Uddannelse}

\experience{Master of Professional Engineering}{The University of Western Australia}{februar 2016--december 2017}{
    \item Specialisering indenfor Elektronikingeniør.
    \item Speciale: udviklede et computersyn system for en fysisk selvkørende bil testsystem i C++. % Results led to a conference paper.
    \item Opnået den højeste gennemsnitskarakter i min årgang på 298 studerende (88.6 / 100. GPA: 6.94 / 7).
}

\experience{Bachelor of Science}{The University of Western Australia}{februar 2013--december 2015}{
    \item Majors: Engineering Science (Electronic) \& Management
    \item Opnået den højeste gennemsnitskarakter i min årgang (90.6 / 100. GPA: 6.96 / 7).
}




\newpage

\section{Publikationer}

\noindent
\begin{tabular}{p{0.05\textwidth}p{0.9\textwidth}}
    \textbf{2024} & \textbf{A. Hurst}, D. E. Lucani, and Q. Zhang, “PairwiseHist: Fast, Accurate, and Space-Efficient Approximate Query Processing with Data Compression,” \textit{International Conference on Very Large Data Bases} (VLDB), 2024. \\

     & \textbf{A. Hurst}, D. E. Lucani, and Q. Zhang, “GreedyGD: Enhanced Generalized Deduplication compression for IoT applications,” \textit{IEEE Transactions on Industrial Informatics}, 2024. \\

    \textbf{2022} & \textbf{A. Hurst}, D E. Lucani, I. Assent, and Q. Zhang, “Glean: Generalized deduplication enabled approximate edge analytics,” \textit{IEEE Internet of Things Journal}, 2022. \\

    \textbf{2021} & \textbf{A. Hurst}, D. E. Lucani, I. Assent, and Q. Zhang, ``Direct analytics of generalized deduplication compressed IoT data,'' \textit{IEEE Global Communications Conference} (GLOBECOM), 2021. \\

    \textbf{2018} & R. Barker, \textbf{A. Hurst}, R. Shrubsall, G. M. Hassan and T. French, "A Low-Cost Hardware-in-the-Loop Agent-Based Simulation Testbed for Autonomous Vehicles," \textit{IEEE/ASME International Conference on Advanced Intelligent Mechatronics}, 2018. \\

    \textbf{2017} & \textbf{A. Hurst}, J. A. Izaac, F. Altaf, V. Baltz, P. J. Metaxas, “Reconfigurable magnetic domain wall pinning using vortex-generated magnetic fields,” \textit{Applied Physics Letters}, 2017. \\
\end{tabular}





\section{Præstationer}

\noindent
\begin{tabular}{lp{0.87\textwidth}}
    \textbf{2023} & EliteForsk-rejsestipendium. \textcolor{subduedcolour}{Uddannelses- og Forskningsministeriet uddeler hvert år op til 20 stipendier på 200.000 kr. til ph.d studerende til støtte for længerevarende studieophold. Kandidater skal være indstillet af deres institut.} \\

    \textbf{2022} & Betsod Prøve i Dansk 3 (PD3) med karakteren 12. \\

    \textbf{2019} & Industrial Products Insights and Solutions (Bronze). \textcolor{subduedcolour}{Tildelt af IBM til konsulenter baserede på erfaring og bidrag til kunder indenfor industrisektoren.} \\

    \textbf{2016} & Institution of Engineering and Technology Prize in Communication Systems. \textcolor{subduedcolour}{Tildelt den studerende, der opnået den højeste karakter i faget. AU\$300 samt et års medlemskab.} \\

    & MRX Technologies Prize in Control Engineering. \textcolor{subduedcolour}{Tildelt den studerende, der opnået den højeste karakter i faget. AU\$1.000.} \\

    & MRX Technologies Prize in Digital and Embedded Systems. \textcolor{subduedcolour}{Tildelt den studerende, der opnået den højeste karakter i faget. AU\$1.000.} \\

    & Monadelphous Prize in Project Management and Engineering Practice,. \textcolor{subduedcolour}{Tildelt den studerende, der opnået den højeste karakter i faget. AU\$1.500.} \\

    & Programmed Professionals Prize in Circuits and Electronic Systems. \textcolor{subduedcolour}{Tildelt den studerende, der opnået den højeste karakter i faget. AU\$2.000.} \\

    \textbf{2015} & Rio Tinto Undergraduate Scholarship in Geology and/or Engineering Science for academic achievement. \textcolor{subduedcolour}{Tildelt baserede på akademisk præstation. AU\$6.000 stipendium samt invitation til to masterclass workshops.} \\

    & Udvalgt til at afholde talen ved min bachelordimissionsceremoni \textcolor{subduedcolour}{(valedictory address).} \\

    & Tesla Medal in Electronic Materials and Devices. \textcolor{subduedcolour}{Tildelt den studerende, der opnået den højeste karakter i faget. % ENSC3014 Electronic Materials and Devices.
    Kontant beløb samt fysisk medalje.} \\

    \textbf{2014} & John \& Robin de Laeter Scholarshop. \textcolor{subduedcolour}{Tildelt for at istandsætte en videnskablig uddannelsesudstilling på The Gravidt Discovery Centre museum.}
\end{tabular}



\end{document}

